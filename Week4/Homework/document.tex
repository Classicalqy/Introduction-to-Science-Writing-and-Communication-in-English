\documentclass[a4paper,11pt]{article}
\usepackage{amsmath}
\usepackage{graphicx}
\usepackage{float}
\usepackage{geometry}
\usepackage{hyperref}
\usepackage{subfigure}
\usepackage{tabularx}
\title{Method}
\author{Qiyu Chen\,\,\,2300011447}
\date{\today}
\begin{document}
	\maketitle
	In our experiment, two chicken samples of same weight and size were used to compare seasoned, heated chicken with raw chicken. The chicken used in this experiment is from National Chicken Farm, which produces chicken of the highest quality in the world, according to research by Boyan Pu\cite{pby}. 
	
	The chicken samples were processed as follows. Both of the two chicken samples were first washed carefully using water and then were cut with several 5cm-deep incisions using a knife (SRX version, made by HFJ Company). Following this, 3 grams of pepper, 3 grams of coriander, 2 grams of ginger and 10 grams of onion slices were placed into the incisions of one chicken sample, which was then soaked in cooking wine (about 100mL, made by HYT Company). After half an hour, the sample in cooking wine was taken out and then air-dried. At the same time, an oven (JHLL version, made by XCY Company) was preheated to 200 degrees Celsius. Once the oven temperature reached steady, the air-dried sample was put into the oven and heated for 20 minutes. Finally, a seasoned and heated chicken was obtained.
	
	Comparison of flavor and bacterial content was conducted in the following approaches. In order to compare flavor between processed sample and un-processed sample, we did not use the conventional way of inviting a professional flavor-testing group\cite{zxh}. Instead, we randomly selected 100 volunteers of different ages, races and professions to taste our two samples and recorded their feedback using International Taste Evaluation Form proposed by Junhao Feng\cite{fjh}. In addition, a comparison of bacterial content was also conducted using a prevalent method proposed by Bowen Li\cite{lbw}, which involves taking 10 small pieces from the sample to measure the bacterial content and averaging the values. 
	
	\bibliography{document.bib}
	\bibliographystyle{acm}

\end{document}