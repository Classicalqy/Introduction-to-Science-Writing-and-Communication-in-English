\documentclass[a4paper,12pt]{article}
\usepackage{amsmath}
\usepackage{geometry}
\usepackage{hyperref}
\usepackage{graphicx}
\usepackage{float}
\author{Qiyu Chen\,\, 2300011447}
\title{An Approach to the Preparation of Chicken Using Heat and Flavoring}
\date{\today}
\begin{document}
	\maketitle
	\section{Introduction}
	Chicken preparation techniques are used in a range of applications both in homes and in restaurants. Chicken is easily available and can be locally produced in most areas; in addition, it is easily digested and low in calories.
	Since Dundee’s pioneering work reporting the natural method of chicken preparation (Dundee et al., 2008) in which the chicken was killed and then eaten raw with salt, there have been significant innovations. Much work has been carried out in France in relation to improving the method of slaughtering chickens, whereas in the USA researchers have concentrated on improving the size of the bird. The natural method is widely used since the time required for the process is extremely short; however, some problems remain unsolved. The flavor of chicken prepared using the Dundee method is often considered unpleasant and there is a well-documented risk of bacterial infection resulting from the consumption of raw meat.
	The aim of this study was to develop a preparation method that would address these two problems. In this report, we describe the new method, which uses seasoning to improve the flavor while heating the chicken in order to kill bacteria prior to eating. 
	\section{Methods}
	In our experiment, two chicken samples of the same weight and size were used to compare seasoned, heated chicken with raw chicken. The chicken used in this experiment is from National Chicken Farm, which produces chicken of the highest quality in the world, according to research by Boyan Pu\cite{pby}. 
	
	The chicken samples were processed as follows. Both of the two chicken samples were first washed carefully using water and then cut with several 5cm-deep incisions using a knife (SRX version, made by HFJ Company). Following this, 3 grams of pepper, 3 grams of coriander, 2 grams of ginger, and 10 grams of onion slices were placed into the incisions of one chicken sample, which was then soaked in cooking wine (about 100mL, made by HYT Company). After half an hour, the sample in cooking wine was taken out and then air-dried. At the same time, an oven (JHLL version, made by XCY Company) was preheated to 200 degrees Celsius. Once the oven temperature reached steady, the air-dried sample was put into the oven and heated for 20 minutes. Finally, a seasoned and heated chicken was obtained.
	
	A comparison of flavor and bacterial content was conducted in the following approaches. In order to compare flavor between the processed sample and the un-processed sample, we did not use the conventional way of inviting a professional flavor-testing group\cite{zxh}. Instead, we randomly selected 100 volunteers of different ages, races, and professions to taste our two samples and recorded their feedback using The International Taste Evaluation Form proposed by Junhao Feng\cite{fjh}. Comparison was also conducted with chicken prepared in other ways using the same evaluation form. In addition, a comparison of bacterial content was also conducted using a prevalent method proposed by Bowen Li\cite{lbw}, which involves taking 10 small pieces from the sample to measure the bacterial content and then averaging the values. In the experiment, Bacteria A, Bacteria B, and Bacteria C were mainly measured. We use the equation(14) from Bowen Li's research\cite{lbw}, involving
	\begin{align}
		\text{Bacterial Content X}=\frac{\text{Number of Bacteria X}(\times 10^{10})}{\text{Volume of Chicken}}.
	\end{align}

    During the experiments, it was noticed that the processed chicken was no longer tender and became a little tough. However, according to the interviews with the volunteers, its toughness is acceptable and has no obvious influence on its taste.
	\section{Results}
    \subsection*{Flavor}
    A comparison between the processed chicken and the unprocessed chicken was made. The experiment involved selecting 100 volunteers of different ages, races, and professions. These volunteers tasted the two samples and then filled out the International Taste Evaluation Form\cite{fjh}. The results indicated that the processed chicken received significantly better feedback than the unprocessed chicken (see Fig 1). Across most evaluation indicators, the processed chicken demonstrated substantial advantages over the unprocessed chicken. For instance, the ‘smell’ indicator scored 92 for the processed chicken, compared to 34 for the unprocessed chicken (out of a total score of 100). Another example is the ‘aftertaste’ indicator, which scored 84 for the processed chicken and only 13 for the unprocessed chicken. Overall, the unprocessed chicken had an advantage in only two out of thirty indicators.

    The score of the processed chicken was also compared with chicken treated in other ways, including Dundee's Method\cite{dundee}, Chenhao Ren's Method\cite{rch}, etc(see Fig 2). Although our processed chicken has lower scores in some indicators, it shows better overall scores compared to any other methods. It indicates that our new method of preparing chicken has more comprehensive abilities and no obvious shortcomings.
    \subsection*{Bacterial Content}
    Besides the flavor comparison, the results of the bacterial content comparison are also presented in Fig 3. We measured the bacterial content using a method proposed by Bowen Li\cite{lbw}. The results indicate a significant decrease in the content of Bacteria A, Bacteria B, and Bacteria C after our chicken preparation method. In contrast to the unprocessed chicken, the processed chicken generally exhibits lower bacterial content across all types of bacteria. Additionally, when comparing our method to other chicken preparation techniques, our approach demonstrates competitive performance (as shown in Fig 4). Our processed chicken has lower Bacteria B content than other methods, while maintaining similar bacterial content for Bacteria A, Bacteria C, and other types. These results align with our expectations prior to conducting the experiment, confirming that our method effectively reduces bacterial content.

    \section{Discussion}
    In this study, we have developed a novel method for chicken preparation that involves both heat and flavoring. Subsequently, we conducted a comparative analysis of the flavor and bacterial content between chicken processed using our method and other existing methods. The results are encouraging, indicating that our new approach surpasses any previously known methods.

    Chicken is a staple food worldwide due to its widespread availability and health benefits. Consequently, it is crucial to develop effective chicken preparation methods that enhance flavor while also reducing bacterial content. Our research significantly contributes to this field. In comparison to currently popular methods\cite{rch}\cite{dundee}\cite{fjh}\cite{zy}, our novel method combines both heating and flavoring, resulting in improved efficiency for reducing bacterial content in the raw chicken while enhancing its flavor.

    A minor limitation emerged during our experiments: the processed chicken exhibited slight toughness due to prolonged heating, as mentioned in the Method Section. To address this issue and further enhance the chicken’s flavor, we recommend exploring adjustments to the ratio of heating time versus air-drying time in future research.

    In summary, our innovative method offers a satisfying approach to chicken preparation. There are sufficient reasons to believe that our method will gain much popularity.
	\bibliography{main.bib}
	\bibliographystyle{acm}
\end{document}