\documentclass[a4paper,12pt]{article}
\usepackage{amsmath}
\usepackage{geometry}
\usepackage{hyperref}
\usepackage{graphicx}
\usepackage{float}
\author{Qiyu Chen\,\, 2300011447}
\title{An Approach to the Preparation of Chicken Using Heat and Flavoring}
\date{\today}
\begin{document}
	\maketitle
	\section{Introduction}
	Chicken preparation techniques are used in a range of applications both in homes and in restaurants. Chicken is easily available and can be locally produced in most areas; in addition, it is easily digested and low in calories.
	Since Dundee’s pioneering work reporting the natural method of chicken preparation (Dundee et al., 2008) in which the chicken was killed and then eaten raw with salt, there have been significant innovations. Much work has been carried out in France in relation to improving the method of slaughtering chickens, whereas in the USA researchers have concentrated on improving the size of the bird. The natural method is widely used since the time required for the process is extremely short; however, some problems remain unsolved. The flavor of chicken prepared using the Dundee method is often considered unpleasant and there is a well-documented risk of bacterial infection resulting from the consumption of raw meat.
	The aim of this study was to develop a preparation method that would address these two problems. In this report, we describe the new method, which uses seasoning to improve the flavor while heating the chicken in order to kill bacteria prior to eating. 
	\section{Methods}
	In our experiment, two chicken samples of same weight and size were used to compare seasoned, heated chicken with raw chicken. The chicken used in this experiment is from National Chicken Farm, which produces chicken of the highest quality in the world, according to research by Boyan Pu\cite{pby}. 
	
	The chicken samples were processed as follows. Both of the two chicken samples were first washed carefully using water and then were cut with several 5cm-deep incisions using a knife (SRX version, made by HFJ Company). Following this, 3 grams of pepper, 3 grams of coriander, 2 grams of ginger and 10 grams of onion slices were placed into the incisions of one chicken sample, which was then soaked in cooking wine (about 100mL, made by HYT Company). After half an hour, the sample in cooking wine was taken out and then air-dried. At the same time, an oven (JHLL version, made by XCY Company) was preheated to 200 degrees Celsius. Once the oven temperature reached steady, the air-dried sample was put into the oven and heated for 20 minutes. Finally, a seasoned and heated chicken was obtained.
	
	Comparison of flavor and bacterial content was conducted in the following approaches. In order to compare flavor between processed sample and un-processed sample, we did not use the conventional way of inviting a professional flavor-testing group\cite{zxh}. Instead, we randomly selected 100 volunteers of different ages, races and professions to taste our two samples and recorded their feedback using International Taste Evaluation Form proposed by Junhao Feng\cite{fjh}. In addition, a comparison of bacterial content was also conducted using a prevalent method proposed by Bowen Li\cite{lbw}, which involves taking 10 small pieces from the sample to measure the bacterial content and averaging the values. 
	
	\bibliography{Homework7.bib}
	\bibliographystyle{acm}
\end{document}